\documentclass{article}
\usepackage[english, russian]{babel}
\usepackage{graphicx}
\usepackage{minted}
\usepackage{hyperref}
\usepackage{amsmath}
\usepackage{xcolor}
\include{dependecies}
\author{Автор: Тевелев Артём Дмитриевич (М3111)}
\title{Работа с \LaTeX. Лабораторная работа \textnumero 3}
\date{\today}
\graphicspath{{images}}

\begin{document}

\maketitle
\begin{center}
    {Преподаватель: Жуйков Артём Сергеевич}
\end{center}
\newpage

\tableofcontents
\newpage

\section{Общее описание библиотеки geometric\_lib}

Библиотека предоставляет функции для вычисления площадей и периметров основных геометрических фигур: кругов, квадратов и прямоугольников. Для каждой фигуры представлены две функции — для вычисления площади и периметра.

\section{Программа 1: Круг}

\subsection{Исходный код программы}

\begin{minted}[linenos, frame=single]{python}
import math

def area(r):
    return math.pi * r * r

def perimeter(r):
    return 2 * math.pi * r
\end{minted}

\subsection{Описание логики программы}

1. Функция `area(r)` — вычисляет площадь круга:
   \[
   S = \pi R^2
   \]
   где \( R \) — радиус круга.

2. Функция `perimeter(r)` — вычисляет периметр (длину окружности):
   \[
   P = 2\pi R
   \]
   где \( R \) — радиус круга.

\subsection{Пример использования}

\begin{equation}
\text{area}(5) = 25 \cdot \pi \quad (\approx 78.54)
\end{equation}
\begin{equation}
\text{perimeter}(5) = 10 \cdot \pi \quad (\approx 31.42)
\end{equation}

\section{Программа 2: Квадрат}

\subsection{Исходный код программы}

\begin{minted}[linenos, frame=single]{python}
def area(a):
    return a * a

def perimeter(a):
    return 4 * a
\end{minted}

\subsection{Описание логики программы}

1. Функция `area(a)` — вычисляет площадь квадрата:
   \[
   S = a^2
   \]
   где \( a \) — длина стороны квадрата.

2. Функция `perimeter(a)` — вычисляет периметр квадрата:
   \[
   P = 4a
   \]
   где \( a \) — длина стороны квадрата.

\subsection{Пример использования}

\begin{verbatim}
area(4) = 4 * 4 = 16
perimeter(4) = 4 * 4 = 16
\end{verbatim}

\section{Программа 3: Прямоугольник}
\subsection{Исходный код программы}

\begin{minted}[linenos, frame=single]{python}
def area(a, b):
    return a * b

def perimeter(a, b):
    return 2 * a + 2 * b
\end{minted}

\subsection{Описание логики программы}

1. Функция `area(a, b)` — вычисляет площадь прямоугольника:
   \[
   S = a \times b
   \]
   где \( a \) и \( b \) — длины сторон прямоугольника.

2. Функция `perimeter(a, b)` — вычисляет периметр прямоугольника:
   \[
   P = 2a + 2b
   \]
   где \( a \) и \( b \) — длины сторон прямоугольника.

\subsection{Пример использования}

\begin{verbatim}
area(3, 6) = 3 * 6 = 18
perimeter(3, 6) = 2 * 3 + 2 * 6 = 18
\end{verbatim}

Ссылка на GitHub проект: \textcolor{blue}{\href{https://github.com/Artem-Tev/Lab3-LaTeX}{Лабораторная на Github}}
\end{document}